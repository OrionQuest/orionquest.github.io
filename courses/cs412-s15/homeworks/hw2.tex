\documentclass{article}
\author{}
\usepackage{amscd, amssymb, amsthm, amsmath, graphicx, fancyhdr}
\newtheorem{lemma}{Lemma}
\newtheorem{openquestion}{Open question}
\newtheorem{conjecture}{Conjecture}
\newtheorem{corrolary}{Corollary}
\newtheorem{theorem}{Theorem}
\newtheorem{observation}{Observation}
\newtheorem{prob}{Problem}
\title{CS412: Homework \#2}
\author{Due: Tuesday, February 24th, 2015 (by 11:59PM)}
\date{}

\begin{document}
\maketitle

\noindent Sum of all problems: 120$\%$, Maximum possible score: $100\%$.
\vspace{.2in}

\noindent For the following problems, even if you do not answer a given
question, you are still allowed to use its result in order to answer a
subsequent question.

\begin{enumerate}
\item $[10\%]$
Express the following polynomial in the correct form for evaluation by Horner's
method:
\begin{eqnarray*}
p(t)=5t^3-3t^2+7t-2
\end{eqnarray*}

\item $[30\%]$ How many multiplications are required to evaluate a polynomial $p(t)$ of
degree $n-1$ at a given point $t$
\begin{enumerate}
\item
Represented in the monomial basis?
\item
Represented in the Lagrange basis?
\item
Represented in the Newton basis?
\end{enumerate}

\item
\begin{enumerate}
\item
$[10\%]$ Determine the polynomial interpolant to the data
\begin{table}[h]
\begin{center}
\begin{tabular}{ccccc}
$t$ & $1$ & $2$ & $3$ & $4$ \\
$y$ & $11$ & $29$ & $65$ & $125$
\end{tabular}
\end{center}
\end{table}

using the monomial basis.
\item
$[10\%]$ Determine the Lagrange polynomial interpolant to the same data and show that the
resulting polynomial is equivalent to that obtained in part (a).

\item
$[30\%]$ Compute the Newton polynomial interpolant to the same data using each of the
three methods discussed in class (triangular matrix, incremental interpolation,
and divided differences) and show that each produces the same result as the
previous two methods.
\end{enumerate}

\item
\begin{enumerate}
\item
$[15\%]$ For a given set of data points $t_1,\ldots,t_n$, define the function $\pi(t)$ by
\begin{eqnarray*}
\pi(t)=(t-t_1)(t-t_2)\ldots (t-t_n)
\end{eqnarray*}

Show that
\begin{eqnarray*}
\pi'(t_j)=(t_j-t_1)\ldots (t_j-t_{j-1})(t_j-t_{j+1})\ldots (t_j-t_n)
\end{eqnarray*}

\item
$[15\%]$ Use the result of part (a) to show that the $j$th Lagrange basis function can be
expressed as
\begin{eqnarray*}
l_j(t)=\frac{\pi(t)}{(t-t_j)\pi'(t_j)}
\end{eqnarray*}
\end{enumerate}
\end{enumerate}

\end{document}
