\documentclass{article}
\author{}
\usepackage{amscd, amssymb, amsthm, amsmath, graphicx, fancyhdr}
\newtheorem{lemma}{Lemma}
\newtheorem{openquestion}{Open question}
\newtheorem{conjecture}{Conjecture}
\newtheorem{corrolary}{Corollary}
\newtheorem{theorem}{Theorem}
\newtheorem{observation}{Observation}
\newtheorem{prob}{Problem}
\title{CS412: Homework \#1}
\author{Due: Tuesday, February 10th, 2015 (in class)}
\date{}

\begin{document}
\maketitle

\noindent Sum of all problems: 120$\%$, Maximum possible score: $100\%$.
\vspace{.2in}

\noindent For the following problems, even if you do not answer a given
question, you are still allowed to use its result in order to answer a
subsequent question.

\begin{enumerate}
\item
$[20\%]$ Each of the following examples describes a fixed point iteration and a nonlinear
equation. In each case, \emph{assuming that the fixed point iteration will
converge}, show that the limit is a solution of the respective equation.
\begin{enumerate}
\item
The iteration $x_{k+1}=\frac{ax_k^2-c}{2ax_k+b}$ and the equation $ax^2+bx+c=0$.
\item
The iteration $x_{k+1}=\frac{3x_k^2+a}{4x_k}$ and the equation $x^2-a=0$.
\end{enumerate}

\item
$[30\%]$ Consider the following pairs, each containing one nonlinear equation and
one iterative procedure:
\begin{enumerate}
\item
The equation $e^x=x+2$ and the iterative method
\begin{eqnarray*}
x_{k+1}=e^{x_k}-2
\end{eqnarray*}
\item
The equation $x^3=x^2+1$ and the iterative method
\begin{eqnarray*}
x_{k+1}=\frac{x_k}{1+x_k^2}
\end{eqnarray*}
\end{enumerate}
For each case, examine if the given iterative procedure is an effective solution
technique for the respective equation. In order for the method to be effective,
it needs to (a) be guaranteed to converge and (b) converge to a solution of the
given equation.

\item
$[30\%]$ Use Newton's method to generate an iterative process that converges to
the following values:
\begin{enumerate}
\item
Create an iterative procedure that computes the cube root $\sqrt[3]{a}$ of a
given number $a$. You are \textbf{not allowed} to use roots in the formula.
[Hint: The cube root is the solution of $x^3-a=0$.]
\item
Create an iterative procedure that computes the natural logarithm $\log a$ of a
given number $a$. You are \textbf{not allowed} to use logarithms in the formula
(exponentials are ok).
\item
Create an iterative procedure that computes the arc-tangent $\arctan(x)$ of a
given number $x$ (remember, the arc-tangent of $x$ is the angle whose tangent
equals $x$). You are \textbf{not allowed} to use inverse trigonometric functions
in the formula (normal trigonometric functions such as $\sin$, $\cos$, $\tan$,
etc., are ok).
\end{enumerate}

\item
$[40\%]$ Consider the following procedure for solving the nonlinear equation
$f(x)=0$:
\begin{itemize}
\item
Start with an initial guess $x_0$.
\item
For $k=0,1,2,\ldots$ do the following:
\begin{itemize}
\item
Compute the value that standard Newton's method would provide, and call it
$\hat x_{k+1}$, i.e.,
\begin{eqnarray*}
\hat x_{k+1}=x_k-\frac{f(x_k)}{f'(x_k)}
\end{eqnarray*}
\item
Compute the next approximation $x_{k+1}$ by averaging $x_k$ and $\hat x_{k+1}$,
i.e.,
\begin{eqnarray*}
x_{k+1}=\frac{x_k+\hat x_{k+1}}{2}
\end{eqnarray*}
\end{itemize}
\end{itemize}
\begin{enumerate}
\item
$[10\%]$ Show that if this method converges, it will converge to a solution of
$f(x)=0$.
\item
$[15\%]$ Show that this method converges under the same conditions as Newton's
method.
\item
$[15\%]$ Determine the order of convergence of this method.
\end{enumerate}
\end{enumerate}

\end{document}
