\documentclass{article}
\author{}
\usepackage{amscd, amssymb, amsthm, amsmath, graphicx, fancyhdr}
\newtheorem{lemma}{Lemma}
\newtheorem{openquestion}{Open question}
\newtheorem{conjecture}{Conjecture}
\newtheorem{corrolary}{Corollary}
\newtheorem{theorem}{Theorem}
\newtheorem{observation}{Observation}
\newtheorem{prob}{Problem}
\title{CS412: Homework \#3}
\author{Due: Tuesday, March 24th, 2015 (by 11:59PM)}
\date{}

\begin{document}
\maketitle

\noindent Sum of all problems: 120$\%$, Maximum possible score: $100\%$.
\vspace{.2in}

\begin{enumerate}
\item $[40\%]$ Prove the following properties:
\begin{enumerate}
\item
Show that for any vector $x\in\mathbb R^n$, the following inequalities hold:
\begin{eqnarray*}
||x||_\infty\leq ||x||_1\leq n||x||_\infty \\
||x||_\infty\leq ||x||_2\leq \sqrt{n}||x||_\infty
\end{eqnarray*}
\item
Assume that positive constants $c_1,c_2$ exist, such that for any $x\in\mathbb
R^n$
\begin{eqnarray*}
c_1||x||_a\leq ||x||_b\leq c_2||x||_a
\end{eqnarray*}
Here, $||\cdot||_a$ and $||\cdot||_b$ are simply two different vector norms.
Show that in this case, we can also find positive constants $d_1,d_2$ such that
\begin{eqnarray*}
d_1||M||_a\leq ||M||_b\leq d_2||M||_a
\end{eqnarray*}
for any \emph{matrix} $M\in\mathbb R^{n\times n}$. The norms in the last
expression are the matrix norms induced from the respective vector norms.
\end{enumerate}
\item
$[20\%]$ Let
\begin{eqnarray*}
A=\left[
\begin{array}{cc}
1 & 1+\varepsilon \\
1-\varepsilon & 1
\end{array}
\right]
\end{eqnarray*}
\begin{enumerate}
\item
What is the determinant of $A$?
\item
In single-precision arithmetic, for what range of values of $\varepsilon$ will
the computed value of the determinant be zero?
\item
What is the $LU$ factorization of $A$?
\item
In single-precision arithmetic, for what range of values of $\varepsilon$ will
the computed value of $U$ be singular?
\end{enumerate}
\newpage
\item
$[20\%]$ Prove the following, where $A,U,V$ are $n\times n$ matrices and $u,v$
are $n\times1$ vectors:
\begin{enumerate}
\item
The Sherman-Morrison formula:
\begin{eqnarray*}
(A-uv^T)^{-1}=A^{-1}+A^{-1}u(1-v^TA^{-1}u)^{-1}v^TA^{-1}
\end{eqnarray*}
\emph{Hint:} Multiply both sides by $(A-uv^T)$.
\item
The Woodbury formula:
\begin{eqnarray*}
(A-UV^T)^{-1}=A^{-1}+A^{-1}U(I-V^TA^{-1}U)^{-1}V^TA^{-1}
\end{eqnarray*}
\emph{Hint:} Multiply both sides by $(A-UV^T)$.
\end{enumerate}
\item $[40\%]$ Prove the following two statements:
\begin{enumerate}
\item
The product of two lower triangular matrices is lower triangular.
\item
The inverse of a nonsingular lower triangular matrix is lower triangular.
\end{enumerate}
\end{enumerate}

\end{document}
